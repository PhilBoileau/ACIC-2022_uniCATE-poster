\documentclass[20pt,a1paper,landscape]{tikzposter}

% required packages
\usepackage{amsmath}
\usepackage{amsfonts}
\usepackage{graphicx}

%% Available themes: see also
%% https://bitbucket.org/surmann/tikzposter/downloads/themes.pdf
\usetheme{Default}
\colorlet{backgroundcolor}{white}

% title information
\title{A Flexible Approach to Predictive Biomarker Discovery}
\author{Philippe Boileau$^1$, Nina Ting Qi$^2$, Mark J. van der Laan$^1$
  Sandrine Dudoit$^1$, Ning Leng$^2$}
\institute{$^1$University of California, Berkeley; $^2$Genentech Inc.}

% dictate default block options
\newcommand{\myblock}[2]{\block[titleinnersep=5mm, linewidth=1mm]{#1}{#2}}


\begin{document}

% title and graphics
\maketitle[width=22in]
\node[anchor=west] at (TP@title.west) {\includegraphics[width=10cm]{logos/cal}};
\node[anchor=east] at (TP@title.east) {\includegraphics[width=10cm]{logos/cal}};

\begin{columns}

  \column{0.333}

  \myblock{Background}{This is the background.}

  \myblock{Variable Importance Parameter}{
    Consider $n$ identically and independently distributed (i.i.d) full-data
    random vectors $X = (W, A, Y^{(0)}, Y^{(1)}) \sim P_X$.

    \begin{itemize}
      \item $W$: A $p$-length random vector of centered pretreatment biomarkers
        with nonzero variance.
      \item $A$: A random binary indicator of treatment assignment.
      \item $Y^{(0)}, Y^{(1)}$: Continuous potential outcomes under assignment
        to the control and treatment allocations, respectively.
    \end{itemize}

    Our causal variable importance parameter is
    $\Psi^F(P_X) = (\Psi_1^F(P_X), \ldots, \Psi_p^F(P_X))$, where
    \begin{equation*}
      \Psi_j^F(P_X) \equiv \frac{
        \mathbb{E}_{P_X}\left[\left(Y^{(1)}-Y^{(0)}\right)W_j\right]
        }{\mathbb{E}_{P_X}\left[W_j^2\right]}.
    \end{equation*}

    Given access instead to $n$ i.i.d. censored random observations $O = (W, A,
    Y)$ where $Y = AY^{(1)} + (1-A)Y^{(0)}$, $\Psi^F(P_X)$ is identified under
    the assumptions of unmeasured confounding and positivity by $\Psi(P_0) =
    (\Psi_1(P_0), \ldots, \Psi_p(P_0))$. Here,
    \begin{equation*}
      \Psi_j(P_0) \equiv \frac{
        \mathbb{E}_{P_0}\left[\left(\bar{Q}_0[A=1,W]-
        \bar{Q}_0[A=0,W]\right)W_j\right]
        }{\mathbb{E}_{P_0}\left[W_j^2\right]},
    \end{equation*}
    where $\bar{Q}_0[A,W] = \mathbb{E}_{P_0}[Y|A,W]$.
  }

  \column{0.334}
  
  \myblock{Inference}{
    Let $g_0(W) = \mathbb{P}[A=1|W]$, and let $\hat{g}$ and $\hat{\bar{Q}}$ be
    estimators of $g_0$ and $\bar{Q}_0$, respectively. Define the Augmented
    Inverse Probability Weighted outcome difference as
    \begin{align*}
      T(O)
      & \equiv \left(\frac{I(A=1)}{\hat{g}(W)}
        - \frac{I(A=0)}{1 - \hat{g}(W)}\right)
        \left(Y-\hat{\bar{Q}}(A,W)\right) \\
      & \qquad + \hat{\bar{Q}}(1,W)-\hat{\bar{Q}}(0, W).
    \end{align*}

    We derive from the efficient influence function of $\Psi_j(P_0)$, $D_j(O)$,
    the double-robust one-step estimator
    \begin{equation*}
      \hat{\Psi}_j(P_n) \equiv \frac{\sum_{i=1}^n T(O_i) W_{ij}}
      {\sum_{i=1}^n W_{ij}^2},
    \end{equation*}
    where $\sum_i W_{ij} = 0$ for all $j$ and $P_n$ is the empirical
    distribution.

    If $\hat{g}$ and $\hat{\bar{Q}}$ are trained using sample splitting
    techniques, and we assume that $\lVert \hat{g} - g_0 \rVert_2 \lVert
    \hat{\bar{Q}} - \bar{Q}_0\rVert_2 = o_p(n^{-1/2})$, then
    \begin{equation*}
      \sqrt{n}\left(\hat{\Psi}_j(P_n) - \Psi_j(P_0)\right)
      \overset{D}{\rightarrow} N\left(0,
      \mathbb{V}_{P_0}\left[D_j(O)\right]\right).
    \end{equation*}
  }


  \myblock{Simulation Study}{Here are the results of our simulation study.}

  \column{0.333}

  \myblock{Clinical Trial Application}{IMmotion 151 heatmaps.}

  \myblock{Conclusions}{Here are the important takeaways.}

  \begin{subcolumns}

    \subcolumn{0.5}
    \myblock{References}{List of references.}

    \subcolumn{0.5}
    \myblock{Funding}{
      {\small PB gratefully acknowledges the support of the National Science and
       Engineering Research Council of Canada and the Fonds de recherche du
       Qu\'{e}bec -- Nature et technologies.}
    }

  \end{subcolumns}

\end{columns}

\end{document}
